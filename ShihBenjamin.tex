% LaTeX resume using res.cls
\documentclass{res}
%\usepackage{helvetica} % uses helvetica postscript font (download helvetica.sty)
%\usepackage{newcent}   % uses new century schoolbook postscript font  
\topmargin=-0.5in %length of margin at the top of the page (1 inch added by default)
\headheight=-1pt %1in margins at top and bottom (1 inch is added to this value by default)
\setlength{\textheight}{11.0in}

\addtolength{\oddsidemargin}{-.475in}
\addtolength{\evensidemargin}{-.475in}
\addtolength{\textwidth}{0.75in}

\begin{document} 

\begin{center}
{\Large \textbf{Benjamin Shih}}
\end{center}
\vspace{-0.2in}
benshih@cmu.edu (301)758-0826 \hfill 5032 Forbes Avenue SMC 6051
\\*United States Citizen \hfill Pittsburgh, PA 15213
\\*https://github.com/benshih \hfill http://ben-shih.com


\vspace{-0.1in}
\section{EDUCATION}
\vspace{0.05in} 
 
Carnegie Mellon University \hfill Pittsburgh, PA
\\*B.S. in Electrical and Computer Engineering \hfill August 2009 - May 2013
\\*GPA: 3.52/4.00, Major GPA: 3.59/4.00

\vspace{-0.1in}
\section{SKILLS}
\vspace{0.05in}
{\bf Software: } MATLAB, Eagle, LaTeX, Cadence, SolidWorks, ProTools, IgorPro
\\*{\bf Electronics: } soldering, oscilloscope, function generator, analog/digital multimeter, circuit simulation, pcb\\
\indent board design, sensor design, microcontroller programming
\\*{\bf Coding: } Java, C, Python, R, x86 Assembly, VBA, HTML, SystemVerilog
\\*{\bf Languages: } English (proficient), Mandarin Chinese (speaking), Spanish (basic), Japanese (basic)

\vspace{-0.1in}
\section{COURSEWORK} 
\vspace{0.05in} 
 {\bf In progress:} Mechatronic Design, Gadgetry, Kinematics/Dynamics/Control, Psychology of Music
 \\*{\bf Completed:} Bio-inspired Robotics, Robot Kinematics, Machine Learning, Controls, Welding, Sensor Systems, Data Structures and Algorithms, Computer Systems, Microelectronics, Graph Theory, Electromagnetics, Noisy Signal Processing, Sound Recording

\vspace{-0.1in}
\section{EMPLOYMENT}
\vspace{0.05in} 
 
 {\bf Signal Processing Department, Carnegie Mellon University} \hfill Pittsburgh, PA
    \\*\emph{Undergraduate Researcher} \hfill May 2012 - present
\begin{itemize}
    \item Apply spectral graph theory to big data. Chunk and process $\sim$20,000 nodes using MATLAB.
    \item Analyze Frobenius norms of Laplacians and adjacency matrices and apply error minimization via matrix perturbation theory and approximations. Reduced error by 2 orders of magnitude.
\end{itemize}
    
    
 {\bf Electrical and Computer Engineering Department, Carnegie Mellon University} \hfill Pittsburgh, PA
    \emph{18-320 Microelectronic Circuits Teaching Assistant} \hfill August 2012 - present
\begin{itemize}
    \item Guide $\sim$30 students through amplifier design (analog) and transistor layouts in Cadence (digital). Lead two 3 hour/week lab sections. 
\end{itemize}     
    \emph{18-290 Signals and Systems Teaching Assistant}\hfill August 2011 - December 2011
\begin{itemize}
    \item Guided $\sim$30 students through various MATLAB activities related to introductory signal processing, including audio/speech processing and specgram analysis. Managed one 3 hour/week lab section. 
\end{itemize}
 
    {\bf NanoJapan, Rice University} \hfill Houston, TX
    \\*\emph{Undergraduate Researcher} \hfill May 2011 - August 2011
\begin{itemize}
    \item Analyzed the vibrational and rotational modes of $C_{60}$ nanocars via Raman spectroscopy.
    \item Worked in a cross-cultural research setting alongside $\sim$40 Japanese graduate students.
    \item Delivered poster presentation at International Symposium on Terahertz Nanoscience (TeraNano) at Osaka University, Japan in November 2011.
\end{itemize}
    
      
\vspace{-0.1in}
\section{PROJECTS} 
\vspace{0.05in} 
    {\bf Fluxgate Magnetometer Sensor } \hfill January 2012 - May 2012
\begin{itemize}
    \item Worked with peer to create MATLAB models to simulate fluctuations in Earth$'$s magnetic field due to perturbations by objects of varying magnetic strength and position/distance. 
    \item Presented device results as technical report. Performed literature reviews for classmates.
\end{itemize}

    {\bf Line-Following Mobile Robot} \hfill October 2011 - April 2012
\begin{itemize}
    \item Worked with peer to create simple scheduler for pulsing motors and reading sensors.
    \item Handmade components: plexiglass chassis, two-link joint for front wheel steering, wheel encoders using black/white tape and infrared sensors, H-bridge for motor control, infrared sensor array for line detection. 
    \item Programmed PIC18F25K22 using C/assembly in MPLabX for controlling steering and monitoring sensors. 
\end{itemize}
 
   
\vspace{-0.1in}
    
\section{HONORS} 
\vspace{0.05in} 
Small Undergraduate Research Grant, Carnegie Mellon University (\$500) \hfill November 2011
\\*NanoJapan NSF International Research Experience for Undergraduates Program \hfill February 2011
\\*NIST Undergraduate Research Fellowship Program \hfill March 2010
\\*Intel Science Talent Search, Semifinalist \hfill January 2009

\end{document}